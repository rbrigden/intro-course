\documentclass{article}
\pagenumbering{gobble}
\begin{document}

\centerline{ \large Course Proposal \n Hack CMMC  }

\centerline{\sc An Introduction to Computer Science and Ethical Hacking}
\vspace{.5pc}
\centerline{\sc December 27, 2014}
 
 
\vspace{0.5pc}
\noindent Now that exam week is over... we can begin our Real World Design Challenge journey! Before we start thinking about this year's challenge, there are a couple things we need to set up and prepare. 

\begin{enumerate}
\item Team Roles
\item Google Docs
\item \LaTeX
\item Hardware and Software
\end{enumerate}

\noindent {\bfseries Team Roles:}
As the challenge strives to incorporate ``real world" engineering and design workflow, team members are each designated a role to play. With our seven-member team, there are seven positions available (duh) as following: 4 x Engineer, CAD Designer, Business Manager and Project Manager. Please note, however, that these positions are very fluid and you will be assigned to work both within and outside your job description. Please bring your preference for team role to the next team meeting.

\begin{itemize}
\item \emph{Engineer:} The engineers will research and provide the theory behind all design aspects. Beyond extrapolation and interpretation of external data, engineers will have to substantiate their findings and proposals through calculation and logical dedution. The engineers are the core of the team.
\item \emph{CAD Designer:} The CAD designer will learn how to utilize three different pieces of engineering software to both design and test the flight vehicle to the specifications provided by the engineers. The first program, MathCAD, is used for optimizing certain design variables. With these optimized assets (wing length, sweep, lift coefficient, etc) the CAD Designer will then port the data to a program called PTC Creo, a three dimensional design software in which the air vehicle will be drawn up. Finally, the design will be tested by the CAD Designer in a virtual "wind tunnel." In the early stages of the competition, the CAD Designer will be responsible for learning how to utilize these software applications.
\item \emph{Business Manager:} As well as trying to build the best system to complete the specified task, the system must also be economical to ``realistically" solve the given problem. The business manager is responsible to understand the direct and eventual costs of all design aspects. As the design becomes finalized, the Business Manager will have to calculate the total cost of the full system (vehicle, infrastructure, support systems, etc) in the context of the objecitve function, as defined in the background and detailed challenge guides available for reference in the Google Drive folder.
\item \emph{Project Manager:} The Project Manager will be responsible for keeping the collaboration organized and production. They will have an eye on all design aspects and must be capable of uniting concepts across disciplines. The Project Manager will be responisble for making sure that work is submitted on time and in an orderly manner. Ultimately, the Project Manager will be responsible for compiling the team's work into the final project submission.
\end{itemize}



\vspace{1pc}
\noindent {\bfseries Google Docs}

Google Documents, via Google Drive, will be our primary collaborative platform. It is fairly self explanatory and all word documents, latex files, draft drawings and CAD model images, among others, will be uploaded here.


\vspace{1pc}
\noindent {\bfseries \LaTeX}

\LaTeX is a typesetting language that allows us to create good looking documents and write professional looking math type with just a little bit of syntax. This document was created using \LaTeX and our final project submission will also be constructed using this format. You will learn the basic \LaTeX syntax \emph{very well} by the end of the competition and it will be very helpful for the rest of your academic career, trust me.I have attached both an intro guide with download links, as well as example source code. Please reach out to me if you have any issues with the installation or having trouble getting your first document started.


\vspace{1pc}
\noindent {\bfseries Hardware \& Software}

There is no specific hardware or software required for working on the competition, however, the CAD Designer will need to use a Windows machine with a minimum of 6GB of RAM. If you do not have a machine that can run Windows but wante to work on the CAD design aspect, do not fret! We can supply the necessary tools.



\vspace{1pc}
\noindent {\bfseries Schedule}

From now until the end of Winter Break, you will be working through the preliminary design process. This means that you will make the most general design decisions and come up with a plausible strategy with which to solve the problem. Once this strategy can be substantiated through calculation, the design can move to the secondary phase. In the best situation, the team will transition between these phases in early January.

\end{document}
